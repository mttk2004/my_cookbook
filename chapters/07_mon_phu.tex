\chapter{Món Ăn Sáng/Phụ}

\textit{Các món ăn đơn giản, nhanh gọn, phù hợp cho bữa sáng hoặc làm món phụ khi cần. Có thể kết hợp linh hoạt với các món chính.}

\vspace{1cm}

% ==================== CƠM CHIÊN ====================
\monan{Cơm Chiên}{
    \item 2-3 bát cơm nguội (cơm để tủ lạnh qua đêm là ngon nhất)
    \item 2 quả trứng
    \item 2-3 chiếc xúc xích (hoặc 100g thịt heo/gà thái nhỏ)
    \item 1/2 cốc rau củ đông lạnh hỗn hợp
    \item 2 tép tỏi
    \item 2-3 muỗng canh nước tương
    \item Hành lá
    \item Muối, tiêu
    \item 2-3 muỗng canh dầu ăn
}{
    \item Cơm nguội bóc tơi, không để vón cục
    \item Xúc xích thái hạt lựu (hoặc thịt thái nhỏ)
    \item Đánh tan trứng vào bát, cho chút muối
    \item Tỏi băm nhuyễn
    \item Hành lá rửa sạch, thái nhỏ
    \item Rau củ đông lạnh rã đông (hoặc dùng luôn)
}{
    \item Bắc chảo lên bếp, cho 1 muỗng dầu vào
    \item Đổ trứng vào, tráng mỏng, chín rồi vớt ra, thái sợi
    \item Cho thêm dầu vào chảo, phi thơm tỏi
    \item Cho xúc xích (hoặc thịt) vào, xào chín
    \item Cho rau củ vào, xào 2-3 phút
    \item Cho cơm nguội vào, dùng đũa hoặc muôi đảo đều, bóc tơi cơm
    \item Nêm nước tương, muối, tiêu, đảo đều
    \item Cho trứng chiên thái sợi vào, đảo đều
    \item Nếm thử gia vị, điều chỉnh cho vừa ăn
    \item Tắt bếp, rắc hành lá lên trên
    \item Dọn ra đĩa, ăn nóng
    \item \textit{Mẹo:} Cơm nguội, khô tơi sẽ chiên ngon và không bị nhão
    \item \textit{Mẹo:} Chiên lửa lớn, đảo nhanh tay để cơm không dính chảo
    \item \textit{Thời gian:} 15-20 phút
}

\vspace{1cm}

\section*{Biến Tấu}

\subsection*{Cơm Chiên Dương Châu}
\begin{itemize}[leftmargin=1.5em]
    \item Thêm tôm khô (ngâm mềm) hoặc tôm tươi
    \item Thêm xá xíu thái hạt lựu (nếu có)
    \item Thêm đậu Hà Lan vào rau củ
    \item Chiên theo công thức Cơm chiên thường
\end{itemize}

\subsection*{Cơm Chiên Cá Mặn}
\begin{itemize}[leftmargin=1.5em]
    \item Dùng cá khô (cá cơm, cá chỉ vàng) chiên giòn, bỏ xương, xé sợi
    \item Thay xúc xích = cá khô
    \item Chiên theo công thức Cơm chiên thường
    \item Giảm lượng nước tương/muối vì cá mặn đã mặn sẵn
\end{itemize}

\subsection*{Cơm Chiên Kim Chi}
\begin{itemize}[leftmargin=1.5em]
    \item Thêm kim chi thái nhỏ (khoảng 1/2 cốc)
    \item Xào kim chi cùng tỏi trước khi cho cơm vào
    \item Có thể thêm thịt heo ba chỉ thái mỏng
    \item Nêm ít hơn vì kim chi đã có vị
\end{itemize}

\vspace{1cm}

\section*{Món Ăn Sáng Khác}

\subsection*{Trứng Cuộn Xúc Xích}
\begin{itemize}[leftmargin=1.5em]
    \item \textbf{Nguyên liệu:} 3 trứng, 3 xúc xích, muối, tiêu, dầu ăn
    \item Chiên xúc xích chín, để riêng
    \item Đánh trứng với muối, tiêu
    \item Đổ trứng vào chảo, tráng mỏng
    \item Đặt xúc xích lên trứng, cuộn lại
    \item Chiên thêm 1-2 phút cho chín đều
    \item Cắt khúc, ăn kèm tương ớt/tương cà
\end{itemize}

\subsection*{Bánh Mì Ốp La}
\begin{itemize}[leftmargin=1.5em]
    \item \textbf{Lưu ý:} Món này cần có bánh mì (không có trong danh sách nguyên liệu ban đầu)
    \item Nếu có bánh mì: Chiên trứng ốp la, kẹp vào bánh mì cùng dưa leo, cà chua, tương ớt
    \item Nhanh gọn cho bữa sáng
\end{itemize}

\subsection*{Salad Rau Trộn}
\begin{itemize}[leftmargin=1.5em]
    \item \textbf{Nguyên liệu:} Xà lách/chân vịt, cà chua, dưa leo, cà rốt, sữa chua
    \item Rửa sạch rau, xé nhỏ
    \item Cà chua, dưa leo thái lát mỏng
    \item Cà rốt bào sợi (hoặc thái mỏng)
    \item Trộn rau vào tô lớn
    \item \textbf{Sốt sữa chua:} Sữa chua + muối + tiêu + chanh + đường, khuấy đều
    \item Rưới sốt lên salad, trộn đều
    \item Ăn ngay (không để lâu vì rau sẽ ra nước)
\end{itemize}

\subsection*{Xúc Xích Chiên Kèm Tương}
\begin{itemize}[leftmargin=1.5em]
    \item Xúc xích cắt khúc hoặc rạch hoa văn
    \item Chiên trên chảo (hoặc nướng air fryer 180°C, 10 phút) cho vàng giòn
    \item Chấm tương cà, tương ớt, hoặc mù tạt
    \item Ăn kèm cơm hoặc ăn vặt
\end{itemize}

\subsection*{Cà Chua Trứng Bác (Scrambled Eggs)}
\begin{itemize}[leftmargin=1.5em]
    \item Cà chua thái múi, xào chín mềm
    \item Đánh trứng, thêm sữa tươi, muối, tiêu
    \item Đổ trứng vào chảo cùng cà chua
    \item Dùng đũa đảo nhẹ liên tục cho trứng vón cục mềm (không để chín khô)
    \item Tắt bếp khi trứng còn hơi ướt (nhiệt dư sẽ làm chín tiếp)
    \item Ăn kèm bánh mì hoặc cơm
\end{itemize}

\vspace{1cm}

\section*{Mẹo Vặt Cơm Chiên}
\begin{itemize}[leftmargin=1.5em]
    \item \textbf{Cơm nguội:} Dùng cơm nguội để tủ lạnh qua đêm, cơm sẽ khô tơi, chiên không bị nhão
    \item \textbf{Cơm nóng:} Nếu chỉ có cơm nóng, trải cơm ra đĩa, phơi cho nguội và hơi khô (khoảng 30 phút)
    \item \textbf{Bóc tơi cơm:} Trước khi chiên, dùng tay hoặc đũa bóc tơi cơm, không để vón cục
    \item \textbf{Lửa lớn:} Chiên cơm trên lửa lớn, đảo nhanh tay sẽ cho cơm thơm, giòn, không dính chảo
    \item \textbf{Chảo nóng:} Chảo phải nóng kỹ trước khi cho nguyên liệu vào
    \item \textbf{Không chen quá nhiều
:} Nếu nhiều cơm, nên chiên từng mẻ cho đều
    \item \textbf{Nước tương cuối:} Rưới nước tương cuối cùng và đảo đều để màu đẹp
    \item \textbf{Thêm hương vị:} Có thể thêm 1/2 muỗng cà phê dầu mè vào cuối để thơm hơn
\end{itemize}

\vspace{1cm}

\section*{Gợi Ý Thực Đơn Bữa Sáng Nhanh}

\begin{enumerate}[leftmargin=1.5em]
    \item \textbf{Bữa sáng 5 phút:}
    \begin{itemize}
        \item Trứng ốp la + Cơm nguội
        \item Xúc xích chiên + Cơm
    \end{itemize}

    \item \textbf{Bữa sáng 10 phút:}
    \begin{itemize}
        \item Trứng chiên + Cơm + Dưa leo
        \item Omelett rau củ + Cơm
    \end{itemize}

    \item \textbf{Bữa sáng 15 phút:}
    \begin{itemize}
        \item Cơm chiên trứng xúc xích
        \item Mì trộn trứng + Xúc xích chiên
    \end{itemize}

    \item \textbf{Bữa sáng lành mạnh:}
    \begin{itemize}
        \item Trứng luộc + Salad rau
        \item Trứng hấp + Rau luộc + Cơm
    \end{itemize}
\end{enumerate}

\vspace{1cm}

\section*{Lời Kết}
Với các nguyên liệu cơ bản và dụng cụ đơ
n giản, bạn đã có thể chế biến được nhiều món ăn ngon, nhanh gọn và lành mạnh. Chìa khóa là:
\begin{itemize}[leftmargin=1.5em]
    \item Chuẩn bị trước (ướp thịt bỏ đông, rửa rau sẵn)
    \item Nắm vững các kỹ thuật cơ bản (xào, luộc, chiên, nướng)
    \item Linh hoạt biến tấu theo nguyên liệu có sẵn
    \item Không ngại thử nghiệm và điều chỉnh khẩu vị theo sở thích
\end{itemize}

\textit{Chúc bạn nấu ăn vui vẻ và ngon miệng!}
