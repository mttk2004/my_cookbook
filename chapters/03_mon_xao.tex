\chapter{Món Xào}

\textit{Các món xào nhanh, giữ được dinh dưỡng và màu sắc tươi ngon của nguyên liệu. Phù hợp cho bữa cơm gia đình hàng ngày.}

\vspace{1cm}

% ==================== THỊT XÀO CÀ CHUA ====================
\monan{Thịt Xào Cà Chua}{2}{20 phút}{
    \item 200g thịt heo (thăn, nạc vai)
    \item 2-3 quả cà chua chín
    \item 1/2 củ hành tây
    \item 2 tép tỏi
    \item 2 muỗng canh nước tương
    \item 1 muỗng cà phê đường
    \item 1 muỗng canh dầu ăn
    \item Tiêu, muối
    \item Hành lá (trang trí)
}{
    \item Thịt rửa sạch, thái lát mỏng, ướp với 1 muỗng canh nước tương, 1/2 muỗng cà phê đường, chút tiêu, để 10 phút
    \item Cà chua rửa sạch, cắt múi cau
    \item Hành tây bóc vỏ, cắt múi cau
    \item Tỏi băm nhuyễn
    \item Hành lá rửa sạch, thái khúc 3cm
}{
    \item Bắc chảo lên bếp, cho dầu vào, đợi nóng
    \item Phi thơm tỏi băm
    \item Cho thịt vào xào trên lửa lớn đến khi thịt chín săn lại, vớt ra
    \item Cho hành tây vào xào thơm 1 phút
    \item Cho cà chua vào xào đến khi cà chua chín mềm, ra nước
    \item Nêm nước tương, đường, muối, tiêu vừa ăn
    \item Cho thịt trở lại, đảo đều 1-2 phút cho ngấm gia vị
    \item Tắt bếp, cho hành lá vào, đảo đều
    \item Dọn ra đĩa, ăn nóng với cơm
    \item \textit{Thời gian:} 15-20 phút
}

% ==================== GÀ XÀO ỚT CHUÔNG ====================
\monan{Gà Xào Ớt Chuông}{1}{15 phút}{
    \item 200g thịt gà (phi lê hoặc đùi)
    \item 1 quả ớt chuông (đỏ hoặc vàng)
    \item 1/2 củ hành tây
    \item 2 tép tỏi
    \item 2 muỗng canh nước tương
    \item 1 muỗng cà phê đường
    \item 1 muỗng canh dầu ăn
    \item Tiêu, muối
}{
    \item Thịt gà rửa sạch, cắt miếng vừa ăn, ướp với 1 muỗng canh nước tương, tiêu, để 10 phút
    \item Ớt chuông rửa sạch, bỏ hạt, cắt miếng vuông
    \item Hành tây bóc vỏ, cắt múi cau
    \item Tỏi băm nhuyễn
}{
    \item Bắc chảo lên bếp, cho dầu vào đun nóng
    \item Phi thơm tỏi băm
    \item Cho gà vào xào trên lửa lớn đến khi chín vàng, vớt ra
    \item Cho hành tây vào xào thơm
    \item Cho ớt chuông vào, xào 2-3 phút (không xào quá lâu để giữ độ giòn)
    \item Nêm nước tương, đường, muối, tiêu
    \item Cho gà trở lại, đảo đều 1-2 phút
    \item Tắt bếp, dọn ra đĩa
    \item \textit{Thời gian:} 15 phút
}

% ==================== THỊT XÀO ĐẬU ====================
\monan{Thịt Xào Đậu}{1}{15 phút}{
    \item 150g thịt heo hoặc gà (thái lát mỏng)
    \item 200g đậu cove hoặc đậu bắp
    \item 3 tép tỏi
    \item 2 muỗng canh nước tương
    \item 1 muỗng cà phê đường
    \item 1 muỗng canh dầu ăn
    \item Tiêu, muối
}{
    \item Thịt rửa sạch, thái lát mỏng, ướp sơ với nước tương, tiêu
    \item Đậu rửa sạch, bẻ bỏ hai đầu, gãy đôi (nếu đậu dài)
    \item Tỏi băm nhuyễn
}{
    \item Đun sôi nước, chần đậu 1-2 phút, vớt ra ngâm nước lạnh (để giữ màu xanh)
    \item Bắc chảo lên bếp, cho dầu vào, phi thơm tỏi
    \item Cho thịt vào xào chín
    \item Cho đậu đã chần vào, xào đều 2-3 phút
    \item Nêm nước tương, đường, muối, tiêu vừa ăn
    \item Đảo đều thêm 1 phút, tắt bếp
    \item Dọn ra đĩa
    \item \textit{Thời gian:} 15 phút
}

% ==================== RAU MỒNG TƠI/MUỐNG XÀO TỎI ====================
\monan{Rau Mồng Tơi/Muống Xào Tỏi}{1}{10 phút}{
    \item 300g rau mồng tơi hoặc rau muống nước
    \item 3-4 tép tỏi
    \item 1.5 muỗng canh dầu ăn
    \item Muối
    \item Nước tương hoặc mắm (tùy thích)
}{
    \item Rau nhặt bỏ lá úa, rửa sạch qua 2-3 nước
    \item Ngâm nước muối loãng 5 phút, vớt ra để ráo
    \item Cắt rau thành khúc 5-6cm
    \item Tỏi bóc vỏ, đập dập hoặc thái lát mỏng
}{
    \item Bắc chảo lên bếp, cho dầu vào, đun nóng
    \item Cho tỏi vào phi thơm (chú ý không để cháy)
    \item Cho rau vào, xào nhanh trên lửa lớn
    \item Nêm muối (hoặc nước tương/mắm) vừa ăn
    \item Xào đều khoảng 2-3 phút đến khi rau chín mềm
    \item Tắt bếp, dọn ra đĩa ngay
    \item \textit{Lưu ý:} Xào rau trên lửa lớn và nhanh tay để rau giữ được màu xanh, không bị nhũn
    \item \textit{Thời gian:} 10 phút
}

% ==================== CẢI THÌA XÀO THỊT ====================
\monan{Cải Thìa Xào Thịt}{1}{15 phút}{
    \item 300g cải thìa baby
    \item 100g thịt heo hoặc gà (thái lát mỏng)
    \item 3 tép tỏi
    \item 1 muỗng canh nước tương
    \item 1 muỗng canh dầu ăn
    \item Muối, tiêu
    \item 1/4 cốc nước (hoặc nước luộc rau)
}{
    \item Cải thìa rửa sạch, cắt khúc 5cm (tách riêng phần lá và phần cuống)
    \item Thịt rửa sạch, thái lát mỏng, ướp sơ với nước tương, tiêu
    \item Tỏi băm nhuyễn
}{
    \item Bắc chảo lên bếp, cho dầu vào, phi thơm tỏi
    \item Cho thịt vào xào chín, vớt ra
    \item Cho phần cuống cải (phần cứng) vào xào trước 1-2 phút
    \item Cho phần lá cải vào, đảo đều
    \item Thêm nước, đun 2-3 phút cho cải mềm
    \item Nêm muối, nước tương vừa ăn
    \item Cho thịt trở lại, đảo đều
    \item Tắt bếp, dọn ra đĩa
    \item \textit{Thời gian:} 15 phút
}

\vspace{1cm}

\section*{Mẹo Vặt Món Xào}
\begin{itemize}[leftmargin=1.5em]
    \item \textbf{Lửa lớn:} Xào trên lửa lớn để rau củ giữ độ giòn, không bị nhũn và ra nhiều nước
    \item \textbf{Chuẩn bị trước:} Sơ chế hết nguyên liệu trước khi bắt đầu xào vì món xào rất nhanh
    \item \textbf{Xào thịt:} Để thịt săn chắc, xào thịt riêng trước, vớt ra, sau đó mới xào rau, cuối cùng cho thịt trở lại
    \item \textbf{Phi tỏi:} Không phi tỏi quá lâu vì sẽ bị cháy và đắng
    \item \textbf{Nêm nếm:} Nên nêm ít trước, nếm thử rồi điều chỉnh vì dễ cứu hơn là mặn quá
    \item \textbf{Chần rau:} Chần rau xanh qua nước sôi rồi ngâm nước lạnh sẽ giữ màu xanh đẹp
    \item \textbf{Dầu ăn:} Không nên dùng quá nhiều dầu, vừa đủ để nguyên liệu không bị khê
\end{itemize}
