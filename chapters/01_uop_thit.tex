\chapter{Cách Ướp Thịt Bỏ Đông}

\textit{Ướp thịt sẵn và bảo quản trong ngăn đông giúp tiết kiệm thời gian nấu nướng hàng ngày. Mỗi lần ướp có thể chia thành nhiều phần nhỏ (200-300g/túi) để dễ rã đông và sử dụng.}

\vspace{1cm}

\section*{Lưu Ý Chung}
\begin{itemize}[leftmargin=1.5em]
    \item \textbf{Bảo quản:} Cho thịt ướp vào túi zip, ép hết không khí, ghi ngày tháng lên túi
    \item \textbf{Thời gian:} Có thể bảo quản 2-3 tuần trong ngăn đông
    \item \textbf{Rã đông:} Chuyển từ ngăn đông sang ngăn mát tủ lạnh trước 1 đêm, hoặc ngâm túi zip vào nước lạnh 30-45 phút
    \item \textbf{Liều lượng:} Mỗi công thức dưới đây cho khoảng 500g thịt
\end{itemize}

\vspace{1cm}

% ==================== GÀ ƯỚP SẢ ====================
\monan{Gà Ướp Sả}{1}{10 phút}{
    \item 500g thịt gà (đùi, phi lê, cánh...)
    \item 3 cây sả (phần gốc màu tím)
    \item 3 tép tỏi
    \item 2 muỗng canh nước mắm
    \item 1 muỗng canh đường
    \item 1-2 quả ớt tươi (tùy khẩu vị)
    \item 1 muỗng canh dầu ăn
    \item 1/2 muỗng cà phê tiêu
}{
    \item Thịt gà rửa sạch, thấm khô, cắt miếng vừa ăn (nếu dùng nguyên đùi/cánh thì có thể để nguyên)
    \item Sả bóc bỏ lớp ngoài, cắt bỏ phần ngọn, chỉ lấy phần gốc màu tím, đập dập và thái nhỏ
    \item Tỏi bóc vỏ, băm nhuyễn hoặc giã
    \item Ớt rửa sạch, thái nhỏ (bỏ hạt nếu không ăn cay)
}{
    \item Cho tỏi băm, sả băm, ớt vào bát lớn
    \item Thêm nước mắm, đường, tiêu, dầu ăn, trộn đều
    \item Cho thịt gà vào, trộn đều để gia vị ngấm khắp
    \item Chia thành từng phần nhỏ (200-300g), cho vào túi zip
    \item Ép hết không khí, ghi ngày tháng, bảo quản ngăn đông
    \item \textit{Cách dùng:} Rã đông và nướng/chiên/xào/nấu soup đều được
}

% ==================== GÀ ƯỚP MẮM RUỐC ====================
\monan{Gà Ướp Mắm Ruốc}{1}{10 phút}{
    \item 500g thịt gà
    \item 2 muỗng canh mắm ruốc
    \item 3 tép tỏi
    \item 1-2 quả ớt tươi
    \item 1 muỗng canh đường
    \item 1 muỗng canh dầu ăn
    \item 1/2 muỗng cà phê tiêu
}{
    \item Thịt gà rửa sạch, thấm khô, cắt miếng vừa ăn
    \item Tỏi bóc vỏ, băm nhuyễn
    \item Ớt rửa sạch, thái nhỏ
}{
    \item Cho mắm ruốc vào bát, thêm đường, khuấy đều cho đường tan
    \item Thêm tỏi băm, ớt, tiêu, dầu ăn, trộn đều
    \item Cho thịt gà vào, trộn đều, ủ ít nhất 15 phút cho ngấm
    \item Chia nhỏ, cho vào túi zip, ép hết không khí
    \item Ghi ngày tháng, bảo quản ngăn đông
    \item \textit{Cách dùng:} Rã đông và nướng/chiên không dầu cho thơm nhất
}

\vspace{1cm}

% ==================== HEO ƯỚP NƯỚC MẮM ====================
\monan{Heo Ướp Nước Mắm}{1}{10 phút}{
    \item 500g thịt heo (thăn, vai, ba chỉ...)
    \item 2 muỗng canh nước mắm
    \item 3 tép tỏi
    \item 1.5 muỗng canh đường
    \item 1/2 muỗng cà phê tiêu đen
    \item 1 muỗng canh dầu ăn
}{
    \item Thịt heo rửa sạch, thấm khô, cắt lát mỏng hoặc miếng vừa ăn
    \item Tỏi bóc vỏ, băm nhuyễn hoặc giã
}{
    \item Cho nước mắm, đường vào bát, khuấy tan đường
    \item Thêm tỏi băm, tiêu, dầu ăn, trộn đều
    \item Cho thịt vào, trộn đều để gia vị bám khắp miếng thịt
    \item Chia nhỏ từng phần, cho vào túi zip, ép không khí
    \item Ghi ngày tháng, bảo quản ngăn đông
    \item \textit{Cách dùng:} Nướng, chiên, xào, kho đều ngon
}

% ==================== HEO ƯỚP KHO QUẸT ====================
\monan{Heo Ướp Kho Quẹt}{1}{15 phút}{
    \item 500g thịt heo
    \item 2-3 muỗng canh kho quẹt
    \item 2 cây sả
    \item 3 tép tỏi
    \item 1-2 quả ớt tươi
    \item 1 muỗng cà phê đường (nếu kho quẹt chưa ngọt)
    \item 1 muỗng canh dầu ăn
}{
    \item Thịt heo rửa sạch, thấm khô, cắt miếng vừa ăn
    \item Sả đập dập, thái nhỏ (lấy phần gốc)
    \item Tỏi băm nhuyễn
    \item Ớt thái nhỏ
}{
    \item Cho kho quẹt vào bát lớn
    \item Thêm tỏi, sả, ớt, đường (nếu dùng), dầu ăn, trộn đều
    \item Cho thịt vào, trộn đều cho gia vị phủ khắp
    \item Ủ 15-30 phút cho ngấm (càng lâu càng ngon)
    \item Chia nhỏ, cho vào túi zip, ép không khí
    \item Ghi ngày tháng, bảo quản ngăn đông
    \item \textit{Cách dùng:} Nướng cho thơm, hoặc xào/kho cùng rau củ
}

\vspace{1cm}

\section*{Mẹo Vặt}
\begin{itemize}[leftmargin=1.5em]
    \item Nên ướp thịt vào cuối tuần khi rảnh, chia nhỏ thành nhiều túi để dùng dần
    \item Khi bỏ vào túi zip, xếp thịt dàn phẳng, ép hết không khí sẽ giúp rã đông nhanh hơn
    \item Có thể dùng ống hút để hút hết không khí trong túi zip (tự tạo chân không đơn giản)
    \item Nếu cần dùng gấp, cho túi zip vào tô nước lạnh (không nên dùng nước nóng vì sẽ làm chín thịt không đều)
\end{itemize}
