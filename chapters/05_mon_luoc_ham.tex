\chapter{Món Luộc/Hầm}

\textit{Các món luộc và hầm giữ trọn dinh dưỡng, ít dầu mỡ, phù hợp cho chế độ ăn lành mạnh. Hương vị thanh đạm, dễ ăn, dễ tiêu hóa.}

\vspace{1cm}

% ==================== GÀ LUỘC GỪNG ====================
\monan{Gà Luộc Gừng}{2}{40 phút}{
    \item 300-400g thịt gà (đùi, cánh, hoặc 1/2 con gà)
    \item 3-4 lát gừng tươi (dày khoảng 0.5cm)
    \item 2 muỗng cà phê muối
    \item 1 cây hành lá (hoặc hành tây)
    \item Nước đủ ngập gà
    \item \textbf{Nước chấm - Muối tiêu chanh:}
    \begin{itemize}
        \item 1 muỗng cà phê muối
        \item 1/2 muỗng cà phê tiêu
        \item 1/2 quả chanh
    \end{itemize}
}{
    \item Gà rửa sạch, chà xát với muối và gừng, rửa lại
    \item Gừng gọt vỏ, thái lát dày
    \item Hành lá rửa sạch, thái khúc 5cm (hoặc hành tây cắt múi cau)
}{
    \item Cho nước vào nồi, đun sôi
    \item Thêm gừng, muối (1 muỗng cà phê) vào nước
    \item Cho gà vào nồi, đảm bảo nước ngập gà
    \item Đun sôi lại, hớt bỏ bọt
    \item Hạ lửa nhỏ, đậy nắp, luộc 25-30 phút (tùy kích thước gà)
    \item Tắt bếp, để gà trong nồi thêm 5-10 phút cho chín đều
    \item Vớt gà ra, ngâm ngay vào tô nước đá (để da giòn, thịt không khô)
    \item Sau 5 phút, vớt ra để ráo, chặt miếng vừa ăn
    \item \textbf{Pha nước chấm:} Trộn muối + tiêu, vắt chanh vào, khuấy đều
    \item Dọn gà ra đĩa, ăn kèm nước chấm muối tiêu chanh
    \item \textit{Lưu ý:} Ngâm nước đá giúp da gà giòn, thịt ngọt hơn
    \item \textit{Thời gian:} 35-40 phút
}

% ==================== THỊT HEO LUỘC ====================
\monan{Thịt Heo Luộc}{1}{35 phút}{
    \item 300g thịt heo (nạc vai, ba chỉ, hoặc thăn)
    \item 2 lát gừng
    \item 1 muỗng cà phê muối
    \item 1 cây hành lá (hoặc 1/2 củ hành tây)
    \item Nước đủ ngập thịt
    \item \textbf{Nước chấm:} Mắm ruốc hoặc kho quẹt
}{
    \item Thịt rửa sạch
    \item Gừng gọt vỏ, đập dập
    \item Hành lá rửa sạch (hoặc hành tây cắt múi)
}{
    \item Cho nước vào nồi, thêm gừng, muối, hành
    \item Đun sôi nước
    \item Cho thịt vào, đun sôi lại
    \item Hớt bỏ bọt nổi lên
    \item Hạ lửa nhỏ, đậy nắp, luộc 25-30 phút (tùy độ dày của thịt)
    \item Dùng đũa đâm thử, nếu nước chảy ra trong là đã chín
    \item Tắt bếp, vớt thịt ra
    \item Ngâm thịt vào nước đá 5 phút (để thịt săn chắc, không khô)
    \item Vớt ra, để ráo, thái lát mỏng
    \item Dọn ra đĩa, chấm mắm ruốc hoặc kho quẹt
    \item \textit{Ăn kèm:} Rau sống (xà lách, dưa leo, rau thơm)
    \item \textit{Thời gian:} 30-35 phút
}

% ==================== CANH CÀ CHUA TRỨNG ====================
\monan{Canh Cà Chua Trứng}{1}{15 phút}{
    \item 2 quả cà chua chín
    \item 2 quả trứng gà
    \item 1 tép tỏi
    \item Hành lá
    \item 3-4 cốc nước
    \item Muối, nước mắm (hoặc bột ngọt)
    \item 1/2 muỗng cà phê đường
    \item 1 muỗng cà phê dầu ăn
}{
    \item Cà chua rửa sạch, cắt múi cau
    \item Trứng đập vào bát, đánh tan
    \item Tỏi băm nhuyễn
    \item Hành lá rửa sạch, thái nhỏ
}{
    \item Bắc nồi lên bếp, cho dầu vào, đun nóng
    \item Phi thơm tỏi băm
    \item Cho cà chua vào, xào 2-3 phút cho cà chua chín mềm
    \item Thêm nước vào, đun sôi
    \item Nêm muối, nước mắm (hoặc bột ngọt), đường vừa ăn
    \item Khi nước sôi, từ từ rót trứng đánh tan vào (rót thành vòng tròn)
    \item Đợi trứng chín nổi lên, tắt bếp
    \item Cho hành lá vào, đảo nhẹ
    \item Múc ra bát, ăn nóng
    \item \textit{Lưu ý:} Rót trứng khi nước đang sôi để trứng nở đẹp
    \item \textit{Thời gian:} 15 phút
}

% ==================== THỊT KHO TRỨNG ====================
\monan{Thịt Kho Trứng}{2}{60 phút}{
    \item 300g thịt heo ba chỉ (hoặc nạc vai)
    \item 4 quả trứng gà
    \item 3 tép tỏi
    \item 2-3 muỗng canh nước mắm
    \item 2 muỗng canh đường
    \item 1 cốc nước dừa (hoặc nước lọc)
    \item Tiêu
    \item 1 muỗng canh dầu ăn
    \item Hành lá (trang trí)
}{
    \item Thịt rửa sạch, cắt miếng vuông vừa ăn (khoảng 3x3cm)
    \item Luộc trứng chín, bóc vỏ
    \item Tỏi bóc vỏ, băm nhuyễn
    \item Hành lá rửa sạch, thái khúc 3cm
}{
    \item Ướp thịt với 1 muỗng canh nước mắm, 1/2 muỗng canh đường, tỏi băm, tiêu, để 15 phút
    \item Bắc nồi lên bếp, cho dầu vào, phi thơm tỏi (nếu còn)
    \item Cho thịt vào, xào săn lại trên lửa lớn
    \item Cho đường vào, đảo đều cho đường tan và tạo màu caramel nhẹ
    \item Thêm nước mắm, nước dừa (hoặc nước lọc), đun sôi
    \item Hạ lửa nhỏ, đậy nắp, kho khoảng 30 phút
    \item Cho trứng vào, kho tiếp 15 phút, đảo đều để trứng ngấm màu
    \item Nêm nếm lại gia vị cho vừa ăn
    \item Kho đến khi nước sệt lại, thịt mềm
    \item Tắt bếp, cho hành lá vào
    \item Dọn ra bát, ăn nóng với cơm
    \item \textit{Lưu ý:} Kho lửa nhỏ để thịt mềm, nước không bị cạn quá nhanh
    \item \textit{Thời gian:} 50-60 phút
}

\vspace{1cm}

\section*{Mẹo Vặt Món Luộc/Hầm}
\begin{itemize}[leftmargin=1.5em]
    \item \textbf{Luộc gà/heo:} Sau khi luộc chín, ngâm ngay vào nước đá sẽ làm thịt săn chắc, da giòn, không bị khô
    \item \textbf{Hớt bọt:} Khi luộc thịt, nhớ hớt bỏ bọt nổi lên để nước trong, thịt không tanh
    \item \textbf{Kiểm tra độ chín:} Dùng đũa đâm vào phần dày nhất, nếu nước chảy ra trong (không hồng) là đã chín
    \item \textbf{Lưu nước luộc:} Nước luộc gà/heo có thể giữ lại nấu cháo, nấu canh rất ngon
    \item \textbf{Kho thịt:} Kho lửa nhỏ, đậy nắp sẽ giúp thịt mềm, nước không bị bay hơi quá nhanh
    \item \textbf{Màu đẹp:} Khi kho, có thể thêm 1 muỗng cà phê nước màu (đường nấu caramel) để thịt có màu nâu đỏ đẹp mắt
    \item \textbf{Thịt mềm hơn:} Có thể thêm 1 lát gừng hoặc vài lát dứa vào khi kho để thịt mềm nhanh hơn
    \item \textbf{Trứng luộc:} Luộc trứng trong nước sôi 10 phút là vừa chín, không bị lòng đen
\end{itemize}

\vspace{1cm}

\section*{Biến Tấu}
\subsection*{Canh Rau Củ Thịt}
\begin{itemize}[leftmargin=1.5em]
    \item Dùng rau củ đông lạnh hỗn hợp thay vì cà chua
    \item Làm theo công thức Canh cà chua trứng nhưng thay cà chua = rau củ
    \item Có thể thêm thịt băm hoặc thịt thái mỏng vào
\end{itemize}

\subsection*{Gà Hầm Gừng}
\begin{itemize}[leftmargin=1.5em]
    \item Làm tương tự Gà luộc gừng nhưng cho thêm nấm (nếu có), cà rốt thái miếng
    \item Sau khi luộc chín, không vớt ra mà tiếp tục hầm lửa nhỏ thêm 15-20 phút
    \item Nêm muối, tiêu, ăn như món canh/soup
\end{itemize}
