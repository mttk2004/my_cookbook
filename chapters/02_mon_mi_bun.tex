\chapter{Món Mì/Bún}

\textit{Các món mì và bún nhanh gọn, dễ làm, phù hợp cho bữa trưa hoặc bữa tối đơn giản. Thời gian chuẩn bị và nấu chỉ khoảng 15-20 phút.}

\vspace{1cm}

% ==================== BÚN/MÌ XÀO THẬP CẨM ====================
\monan{Bún/Mì Xào Thập Cẩm}{
    \item 200g mì gạo hoặc bún khô
    \item 150g thịt heo hoặc gà (cắt lát mỏng)
    \item 1 cốc rau củ đông lạnh hỗn hợp
    \item 2 tép tỏi
    \item 2 muỗng canh nước tương
    \item 1 muỗng canh dầu ăn
    \item 1/2 muỗng cà phê đường
    \item Tiêu, muối (vừa đủ)
    \item Hành lá (trang trí)
}{
    \item Ngâm mì/bún trong nước ấm 10-15 phút cho mềm, vớt ra để ráo
    \item Thịt cắt lát mỏng, ướp chút nước tương, tiêu
    \item Tỏi băm nhuyễn
    \item Rã đông rau củ (hoặc dùng luôn nếu muốn)
}{
    \item Bắc chảo lên bếp, cho dầu vào, phi thơm tỏi
    \item Cho thịt vào xào chín, vớt ra
    \item Cho rau củ vào xào 2-3 phút
    \item Cho mì/bún đã ngâm vào, đảo đều
    \item Nêm nước tương, đường, tiêu, muối vừa ăn
    \item Cho thịt trở lại, đảo đều thêm 1-2 phút
    \item Tắt bếp, rắc hành lá, dọn ra đĩa
    \item \textit{Thời gian:} 15-20 phút
}

% ==================== MÌ TRỘN TRỨNG ====================
\monan{Mì Trộn Trứng}{
    \item 200g mì gạo khô (hoặc bún khô)
    \item 2 quả trứng
    \item 2 muỗng canh nước tương
    \item 1 muỗng cà phê dầu mè (nếu có, không có thì dùng dầu ăn)
    \item 1 muỗng cà phé đường
    \item 1 tép tỏi
    \item Hành lá thái nhỏ
    \item Tiêu
    \item Dầu ăn để chiên trứng
}{
    \item Luộc mì/bún theo hướng dẫn trên bao bì (thường 3-5 phút), vớt ra rổ, xả nước lạnh, để ráo
    \item Tỏi băm nhỏ
    \item Hành lá rửa sạch, thái nhỏ
}{
    \item Đánh trứng vào bát, thêm chút muối, tiêu, đánh tan
    \item Bắc chảo lên bếp, cho dầu vào, chiên trứng ốp la (hoặc trứng tráng mỏng rồi cuộn lại thái sợi)
    \item Vớt trứng ra đĩa
    \item Trộn nước tương + đường + dầu mè (hoặc dầu ăn) + tiêu + tỏi băm trong bát nhỏ
    \item Cho mì/bún đã luộc vào bát lớn, chan hỗn hợp nước tương lên, trộn đều
    \item Đặt trứng ốp la lên trên, rắc hành lá
    \item \textit{Ăn kèm:} Tương ớt, tương cà tùy thích
    \item \textit{Thời gian:} 10-15 phút
}

% ==================== BÚN CHẢ GIÒ RẾ ====================
\monan{Bún Chả Giò Rế}{
    \item 200g bún khô
    \item 3-4 chiếc xúc xích
    \item Xà lách/chân vịt
    \item Dưa leo
    \item Cà chua
    \item Rau thơm (nếu có)
    \item Dầu ăn để chiên
}{
    \item Luộc bún khô trong nước sôi 3-5 phút, vớt ra rổ, xả nước lạnh, để ráo
    \item Xúc xích cắt miếng xiên hoặc cắt lát mỏng
    \item Xà lách rửa sạch, xé nhỏ
    \item Dưa leo, cà chua rửa sạch, thái lát mỏng
}{
    \item Bắc chảo lên bếp, cho dầu vào
    \item Chiên xúc xích đến khi vàng giòn, vớt ra để ráo dầu
    \item \textbf{Pha nước mắm chua ngọt:}
    \begin{itemize}
        \item 2 muỗng canh nước mắm
        \item 2 muỗng canh đường
        \item 3 muỗng canh nước lọc (hoặc nước ấm)
        \item 1 muỗng canh nước cốt chanh (hoặc giấm)
        \item 1 tép tỏi băm, ớt thái nhỏ
        \item Khuấy tan đường
    \end{itemize}
    \item Xếp bún ra bát/đĩa, cho rau sống, dưa leo, cà chua lên trên
    \item Xếp xúc xích chiên giòn lên cùng
    \item Chan nước mắm chua ngọt lên, trộn đều khi ăn
    \item \textit{Thời gian:} 15 phút
}

\vspace{1cm}

\section*{Mẹo Vặt}
\begin{itemize}[leftmargin=1.5em]
    \item \textbf{Luộc mì/bún:} Không nên luộc quá lâu vì sẽ bị nhũn. Nên thử nếm sau 3 phút để kiểm tra độ chín
    \item \textbf{Xả nước lạnh:} Sau khi luộc nên xả qua nước lạnh để mì/bún không bị dính vào nhau
    \item \textbf{Nước mắm chua ngọt:} Có thể pha sẵn 1 lọ lớn bảo quản trong tủ lạnh, dùng dần trong tuần
    \item \textbf{Thay thế:} Có thể thay thịt bằng thịt đã ướp sẵn từ ngăn đông (rã đông trước)
    \item \textbf{Rau sống:} Rửa sạch rau, ngâm nước muối loãng 5 phút rồi xả lại để đảm bảo vệ sinh
\end{itemize}
