\chapter{Món Trứng}

\textit{Trứng là nguyên liệu dễ kiếm, giá rẻ, giàu dinh dưỡng và chế biến nhanh. Đây là những món ăn cơ bản nhất, phù hợp cho mọi bữa ăn trong ngày.}

\vspace{1cm}

% ==================== TRỨNG ỐP LA ====================
\monan{Trứng Ốp La}{
    \item 2 quả trứng gà
    \item 1-2 muỗng canh dầu ăn
    \item Muối, tiêu (tùy thích)
}{
    \item Không cần sơ chế
}{
    \item Bắc chảo lên bếp, cho dầu vào, đun nóng ở lửa vừa
    \item Đập trứng vào bát nhỏ trước (để kiểm tra trứng tươi)
    \item Cho trứng từ bát vào chảo
    \item Chiên lửa vừa đến khi lòng trắng chín, lòng đỏ còn lỏng (hoặc chín tùy thích)
    \item Rắc muối, tiêu lên trứng
    \item Vớt ra đĩa, ăn nóng với cơm
    \item \textit{Mẹo:} Nếu muốn lòng đỏ chín đều, có thể đậy nắp chảo lại 1-2 phút
    \item \textit{Mẹo:} Cho 1-2 muỗng nước vào chảo rồi đậy nắp sẽ giúp trứng chín đều mà không cần lật
    \item \textit{Thời gian:} 5 phút
}

% ==================== TRỨNG CHIÊN ====================
\monan{Trứng Chiên}{
    \item 2 quả trứng gà
    \item 2 muỗng canh dầu ăn
    \item 1/4 muỗng cà phê muối
    \item Chút tiêu
    \item Hành lá (tùy thích)
}{
    \item Đập trứng vào bát
    \item Hành lá (nếu dùng) rửa sạch, thái nhỏ
}{
    \item Cho muối, tiêu vào bát trứng
    \item Đánh tan trứng bằng đũa (hoặc nĩa) cho đến khi lòng đỏ và lòng trắng hòa quyện
    \item Thêm hành lá thái nhỏ (nếu dùng), trộn đều
    \item Bắc chảo lên bếp, cho dầu vào, đun nóng ở lửa vừa
    \item Đợi dầu nóng, cho trứng vào
    \item Chiên đến khi trứng chín vàng ở đáy, lật mặt kia
    \item Chiên thêm 1-2 phút cho chín đều
    \item Vớt ra đĩa lót giấy thấm dầu
    \item Ăn nóng với cơm hoặc cháo
    \item \textit{Mẹo:} Đánh trứng kỹ sẽ làm trứng chiên mềm, xốp hơn
    \item \textit{Thời gian:} 5-7 phút
}

% ==================== TRỨNG LUỘC ====================
\monan{Trứng Luộc}{
    \item 4 quả trứng gà
    \item Nước (đủ ngập trứng)
    \item Muối tiêu chanh (để chấm)
    \item Hoặc nước tương (để chấm)
}{
    \item Không cần sơ chế
}{
    \item Cho trứng vào nồi, đổ nước lạnh ngập trứng
    \item Bật lửa lớn, đun sôi
    \item Khi nước sôi, hạ lửa vừa, luộc thêm:
    \begin{itemize}
        \item 6-7 phút: lòng đào (lòng đỏ còn chảy)
        \item 10 phút: chín vừa (lòng đỏ chín mềm, chưa khô)
        \item 12-13 phút: chín kỹ (lòng đỏ chín kỹ)
    \end{itemize}
    \item Tắt bếp, vớt trứng ra ngâm ngay vào nước lạnh (hoặc nước đá)
    \item Để nguội 3-5 phút, bóc vỏ
    \item Ăn nguyên hoặc cắt đôi, rắc muối tiêu, vắt chanh
    \item Hoặc chấm nước tương
    \item \textit{Mẹo:} Ngâm nước đá sau khi luộc giúp bóc vỏ dễ dàng, trứng không bị lòng đen
    \item \textit{Thời gian:} 10-15 phút
}

% ==================== TRỨNG HẤP ====================
\monan{Trứng Hấp}{
    \item 3 quả trứng gà
    \item 1/2 cốc sữa tươi (hoặc nước lọc)
    \item 1/4 muỗng cà phê muối
    \item 1 cây hành lá
    \item Chút dầu ăn (để thoa bát)
    \item Tiêu (tùy thích)
}{
    \item Đập trứng vào bát lớn
    \item Hành lá rửa sạch, thái nhỏ
}{
    \item Đánh tan trứng, thêm sữa tươi (hoặc nước), muối, tiêu
    \item Khuấy đều (đừng đánh tạo bọt)
    \item Lọc qua rây để bỏ bọt và cặn (cho trứng hấp mịn)
    \item Thoa chút dầu vào bát/tô sứ chịu nhiệt
    \item Đổ hỗn hợp trứng vào bát
    \item Cho nước vào nồi, đun sôi
    \item Đặt bát trứng lên giá hấp (hoặc úp bát nhỏ lên để đỡ)
    \item Đậy nắp, hấp lửa nhỏ vừa 10-15 phút
    \item Mở nắp, dùng đũa đâm thử, nếu không dính trứng lỏng là đã chín
    \item Tắt bếp, lấy bát ra
    \item Rắc hành lá, chút nước tương lên trên
    \item Ăn nóng
    \item \textit{Mẹo:} Hấp lửa nhỏ để trứng mịn, không bị lỗ chỗ
    \item \textit{Mẹo:} Lọc qua rây giúp trứng hấp mịn màng hơn
    \item \textit{Thời gian:} 20 phút
}

% ==================== OMELETT RAU CỦ ====================
\monan{Omelett Rau Củ}{
    \item 3 quả trứng gà
    \item 1/2 cốc rau củ đông lạnh hỗn hợp (hoặc rau tươi: cà rốt, đậu, bắp...)
    \item 2 muỗng canh sữa tươi
    \item 1/4 muỗng cà phê muối
    \item Tiêu
    \item 1-2 muỗng canh dầu ăn
    \item Hành lá (tùy thích)
}{
    \item Rau củ đông lạnh rã đông (hoặc rau tươi thái nhỏ hạt lựu)
    \item Đập trứng vào bát
    \item Hành lá (nếu dùng) thái nhỏ
}{
    \item Cho trứng, sữa tươi, muối, tiêu vào bát, đánh đều
    \item Thêm rau củ vào hỗn hợp trứng, trộn đều
    \item Bắc chảo lên bếp, cho dầu vào, đun nóng ở lửa vừa
    \item Đổ hỗn hợp trứng rau vào chảo, dàn đều
    \item Chiên lửa nhỏ vừa đến khi mặt dưới chín vàng (khoảng 3-4 phút)
    \item Lật mặt kia (hoặc gấp đôi), chiên thêm 2-3 phút
    \item Vớt ra đĩa
    \item Rắc hành lá lên trên (nếu thích)
    \item Ăn nóng với bánh mì hoặc cơm
    \item \textit{Mẹo:} Có thể thêm phô mai, xúc xích thái nhỏ vào cho đậm đà
    \item \textit{Thời gian:} 10-15 phút
}

\vspace{1cm}

\section*{Mẹo Vặt Món Trứng}
\begin{itemize}[leftmargin=1.5em]
    \item \textbf{Kiểm tra trứng tươi:} Ngâm trứng vào nước, nếu chìm ngang đáy là tươi, nổi lên là để lâu
    \item \textbf{Đập trứng:} Nên đập trứng vào bát nhỏ trước, kiểm tra rồi mới cho vào chảo/bát chính (phòng trứng hỏng)
    \item \textbf{Chiên trứng không dính:} Đợi dầu thật nóng rồi mới cho trứng vào, lửa vừa
    \item \textbf{Trứng ốp la giòn:} Dùng lửa lớn, dầu nhiều sẽ có trứng ốp la giòn xốp; lửa nhỏ, dầu ít sẽ có trứng mềm
    \item \textbf{Luộc trứng dễ bóc:} Ngâm nước đá ngay sau khi luộc xong
    \item \textbf{Trứng hấp mịn:} Hấp lửa nhỏ, lọc qua rây để bỏ bọt
    \item \textbf{Trứng mềm:} Thêm chút sữa tươi vào trứng đánh sẽ làm trứng chiên/hấp mềm hơn
    \item \textbf{Bảo quản:} Trứng để tủ lạnh giữ được lâu hơn, nhưng nên để ở ngăn mát (không đông lạnh)
\end{itemize}

\vspace{1cm}

\section*{Biến Tấu}
\subsection*{Trứng Chiên Xúc Xích}
\begin{itemize}[leftmargin=1.5em]
    \item Xúc xích thái lát mỏng, chiên chín trước, vớt ra
    \item Đánh trứng, cho xúc xích vào, chiên như Trứng chiên thường
\end{itemize}

\subsection*{Trứng Chiên Cà Chua}
\begin{itemize}[leftmargin=1.5em]
    \item Cà chua thái múi, xào chín mềm
    \item Đổ trứng đánh tan vào, đảo đều, chiên chung đến khi trứng chín
    \item Nêm muối, tiêu, hành lá
\end{itemize}

\subsection*{Trứng Luộc Sốt Cà Chua}
\begin{itemize}[leftmargin=1.5em]
    \item Luộc trứng chín, bóc vỏ
    \item Làm sốt cà chua (cà chua xào + tương cà + đường + muối)
    \item Chan sốt lên trứng, ăn nóng
\end{itemize}
