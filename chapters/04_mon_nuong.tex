\chapter{Món Nướng Air Fryer}

\textit{Món nướng bằng nồi chiên không dầu giúp giảm lượng dầu mỡ, thịt vẫn thơm ngon và giòn rụm. Đây là cách chế biến lành mạnh và tiện lợi.}

\vspace{1cm}

\section*{Lưu Ý Khi Nướng Bằng Air Fryer}
\begin{itemize}[leftmargin=1.5em]
    \item \textbf{Làm nóng trước:} Nên làm nóng air fryer 3-5 phút ở 180°C trước khi cho thức ăn vào
    \item \textbf{Không xếp chồng:} Xếp thịt cách nhau, không chồng lên nhau để chín đều
    \item \textbf{Lật giữa chừng:} Nên lật thịt 1-2 lần trong quá trình nướng để chín đều hai mặt
    \item \textbf{Kiểm tra độ chín:} Dùng đũa đâm vào thịt, nếu nước chảy ra trong thì đã chín
    \item \textbf{Nhiệt độ:} Thường dùng 180-200°C cho thịt gà, heo
\end{itemize}

\vspace{1cm}

% ==================== GÀ NƯỚNG SẢ ====================
\monan{Gà Nướng Sả (Air Fryer)}{2}{30 phút}{
    \item 300-400g thịt gà đã ướp sả (từ ngăn đông)
    \item Hoặc nếu ướp tươi:
    \begin{itemize}
        \item 300g thịt gà
        \item 2 cây sả
        \item 2 tép tỏi
        \item 1.5 muỗng canh nước mắm
        \item 1 muỗng cà phê đường
        \item 1 muỗng cà phê dầu ăn
        \item Tiêu, ớt tươi
    \end{itemize}
}{
    \item \textbf{Nếu dùng thịt đông lạnh:}
    \begin{itemize}
        \item Rã đông thịt gà ướp sả (chuyển từ ngăn đông sang ngăn mát trước 1 đêm, hoặc ngâm túi zip vào nước lạnh 30-45 phút)
        \item Vớt thịt ra, để ráo nước
    \end{itemize}
    \item \textbf{Nếu ướp tươi:}
    \begin{itemize}
        \item Sả đập dập, thái nhỏ; tỏi băm; ớt thái nhỏ
        \item Trộn sả, tỏi, ớt, nước mắm, đường, dầu, tiêu
        \item Ướp thịt gà ít nhất 30 phút (càng lâu càng ngấm)
    \end{itemize}
}{
    \item Làm nóng air fryer ở 180°C trong 5 phút
    \item Xếp thịt gà vào rổ air fryer, không xếp chồng lên nhau
    \item Nướng ở 180°C trong 15 phút
    \item Mở air fryer, lật thịt sang mặt kia
    \item Nướng tiếp 10-12 phút ở 180°C (hoặc cho đến khi thịt chín vàng)
    \item Có thể tăng nhiệt lên 200°C trong 2-3 phút cuối để da giòn hơn
    \item Vớt thịt ra, để nguội 2-3 phút
    \item Dọn ra đĩa, ăn nóng với cơm hoặc bún
    \item \textit{Ăn kèm:} Rau sống, nước mắm chua ngọt, cơm/bún
    \item \textit{Thời gian:} 25-30 phút (không kể rã đông)
}

% ==================== HEO NƯỚNG SẢ ====================
\monan{Heo Nướng Sả (Air Fryer)}{2}{30 phút}{
    \item 300-400g thịt heo ướp kho quẹt hoặc ướp sả (từ ngăn đông)
    \item Hoặc nếu ướp tươi:
    \begin{itemize}
        \item 300g thịt heo (thăn, vai, ba chỉ mỏng)
        \item 2 cây sả
        \item 2 tép tỏi
        \item 1.5 muỗng canh nước mắm
        \item 1 muỗng canh đường
        \item 1 muỗng cà phê dầu ăn
        \item Tiêu
    \end{itemize}
}{
    \item \textbf{Nếu dùng thịt đông lạnh:}
    \begin{itemize}
        \item Rã đông thịt heo ướp (chuyển từ ngăn đông sang ngăn mát trước 1 đêm)
        \item Vớt ra, để ráo
    \end{itemize}
    \item \textbf{Nếu ướp tươi:}
    \begin{itemize}
        \item Thịt rửa sạch, cắt lát mỏng (khoảng 0.5cm) hoặc miếng vừa ăn
        \item Sả đập dập, thái nhỏ; tỏi băm
        \item Trộn đều sả, tỏi, nước mắm, đường, dầu, tiêu
        \item Ướp thịt ít nhất 30 phút (hoặc qua đêm trong tủ lạnh)
    \end{itemize}
}{
    \item Làm nóng air fryer ở 180°C trong 5 phút
    \item Xếp thịt vào rổ, để cách nhau (không chồng lên)
    \item Nướng ở 180°C trong 12 phút
    \item Mở air fryer, lật thịt
    \item Nướng tiếp 10 phút ở 180°C
    \item Tăng nhiệt lên 200°C, nướng thêm 3-5 phút cho thịt chín vàng, hơi khê cạnh
    \item Vớt ra, để nguội vài phút
    \item Dọn ra đĩa
    \item \textit{Ăn kèm:} Cơm trắng, rau sống, nước mắm pha, hoặc làm bún thịt nướng
    \item \textit{Thời gian:} 25-30 phút
}

\vspace{1cm}

\section*{Biến Tấu}
\subsection*{Gà/Heo Nướng Mắm Ruốc}
\begin{itemize}[leftmargin=1.5em]
    \item Dùng thịt gà/heo ướp mắm ruốc từ ngăn đông
    \item Rã đông và làm theo các bước nướng tương tự như trên
    \item Nhiệt độ: 180°C, thời gian: 20-25 phút, nhớ lật giữa chừng
    \item Mùi vị đậm đà hơn, thơm mùi mắm ruốc nướng
\end{itemize}

\subsection*{Gà/Heo Nướng Nước Mắm}
\begin{itemize}[leftmargin=1.5em]
    \item Dùng thịt heo ướp nước mắm từ ngăn đông
    \item Rã đông và nướng như công thức trên
    \item Có thể quét thêm chút mật ong hoặc đường ở 5 phút cuối để tạo lớp caramel bóng đẹp
\end{itemize}

\vspace{1cm}

\section*{Mẹo Vặt}
\begin{itemize}[leftmargin=1.5em]
    \item \textbf{Thịt mỏng chín nhanh:} Nếu thái thịt mỏng (0.5cm), thời gian nướng giảm còn 15-20 phút
    \item \textbf{Thịt dày:} Nếu dùng đùi gà nguyên con hoặc sườn heo dày, cần tăng thời gian lên 30-35 phút
    \item \textbf{Kiểm tra độ chín:} Dùng đũa đâm vào phần dày nhất, nếu nước chảy ra trong (không còn hồng) là chín
    \item \textbf{Lót giấy bạc:} Có thể lót giấy bạc dưới rổ air fryer để dễ vệ sinh sau khi nướng
    \item \textbf{Phun dầu:} Nếu thịt hơi khô, có thể phun thêm chút dầu ăn lên bề mặt trước khi nướng
    \item \textbf{Rã đông đúng cách:} Không nên dùng microwave rã đông vì sẽ làm thịt chín không đều. Tốt nhất là chuyển từ ngăn đông sang ngăn mát tủ lạnh trước 1 đêm
    \item \textbf{Nướng cùng rau:} Có thể nướng cùng với ớt chuông, hành tây để có thêm rau ăn kèm
\end{itemize}
