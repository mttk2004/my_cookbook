% ================================================================
% KIẾN THỨC CƠ BẢN VỀ NẤU ĂN (COOKING BASICS)
% ================================================================
% File này chứa các kiến thức cơ bản cần thiết cho người mới bắt đầu

\chapter*{Kiến Thức Cơ Bản}
\addcontentsline{toc}{chapter}{Kiến Thức Cơ Bản}

\section*{Dụng Cụ Cần Thiết}

\subsection*{Dụng Cụ Nấu Nướng}
\begin{itemize}[leftmargin=1.5em]
    \item \textbf{Nồi:} 1 cái nồi nhỏ (1-2 lít) để luộc, nấu canh
    \item \textbf{Chảo:} 1 cái chảo chống dính (đường kính 24-26cm) để xào, chiên
    \item \textbf{Air Fryer:} Nồi chiên không dầu (nếu có) - tiện lợi cho món nướng
    \item \textbf{Vung/Nắp:} Để đậy nồi, chảo
    \item \textbf{Thớt:} Ít nhất 2 cái (1 cho thịt sống, 1 cho rau/thực phẩm chín)
    \item \textbf{Dao:} 1 con dao bếp đa năng (lưỡi dài 15-20cm)
    \item \textbf{Muôi/Xẻng:} Để xào, đảo thức ăn
    \item \textbf{Đũa:} Để gắp, đảo thức ăn
    \item \textbf{Thìa canh/Muỗng cà phê:} Để đong gia vị
    \item \textbf{Rây/Rổ:} Để vớt thức ăn, lọc nước
    \item \textbf{Bát/Tô:} Nhiều kích cỡ để đựng nguyên liệu, pha gia vị
\end{itemize}

\subsection*{Dụng Cụ Bảo Quản}
\begin{itemize}[leftmargin=1.5em]
    \item \textbf{Túi zip:} Để bảo quản thịt ướp, thực phẩm trong ngăn đông
    \item \textbf{Hộp nhựa có nắp:} Để đựng thức ăn thừa, rau củ đã sơ chế
    \item \textbf{Màng bọc thực phẩm:} Để bọc thức ăn
    \item \textbf{Giấy bạc:} Để gói, nướng thức ăn
\end{itemize}

\vspace{1cm}

\section*{Bảng Đo Lường \& Chuyển Đổi}

\subsection*{Đơn Vị Thể Tích}
\begin{itemize}[leftmargin=1.5em]
    \item \textbf{1 muỗng canh (tablespoon)} = 15ml ≈ 3 muỗng cà phê
    \item \textbf{1 muỗng cà phê (teaspoon)} = 5ml
    \item \textbf{1 cốc} = 200-240ml (tùy loại cốc)
    \item \textbf{1 lít} = 1000ml = 4-5 cốc
\end{itemize}

\subsection*{Khối Lượng Ước Lượng}
\begin{itemize}[leftmargin=1.5em]
    \item \textbf{1 muỗng canh đường} ≈ 12-15g
    \item \textbf{1 muỗng canh muối} ≈ 18-20g
    \item \textbf{1 muỗng canh nước mắm} ≈ 15ml
    \item \textbf{1 muỗng canh dầu ăn} ≈ 15ml
    \item \textbf{1 tép tỏi vừa} ≈ 5-7g
\end{itemize}

\subsection*{Nhiệt Độ Air Fryer/Lò Nướng}
\begin{itemize}[leftmargin=1.5em]
    \item \textbf{150-160°C:} Sấy khô, hâm nóng
    \item \textbf{180°C:} Nướng thịt gà, heo (phổ biến nhất)
    \item \textbf{200°C:} Nướng nhanh, làm giòn
    \item \textbf{220-230°C:} Nướng thịt dày, tạo vỏ giòn
\end{itemize}

\vspace{1cm}

\section*{Bảo Quản Thực Phẩm}

\subsection*{Trong Tủ Lạnh (Ngăn Mát - 0-4°C)}
\begin{itemize}[leftmargin=1.5em]
    \item \textbf{Thịt tươi:} 1-2 ngày
    \item \textbf{Rau củ tươi:} 3-7 ngày (tùy loại)
    \item \textbf{Trứng:} 3-5 tuần
    \item \textbf{Thức ăn đã nấu chín:} 3-4 ngày (trong hộp kín)
    \item \textbf{Rau củ đã luộc/xào:} 2-3 ngày
\end{itemize}

\subsection*{Trong Ngăn Đông (-18°C)}
\begin{itemize}[leftmargin=1.5em]
    \item \textbf{Thịt ướp sẵn:} 2-3 tuần (trong túi zip kín)
    \item \textbf{Thịt tươi chưa ướp:} 3-6 tháng
    \item \textbf{Rau củ đông lạnh:} 8-12 tháng
    \item \textbf{Thức ăn nấu chín:} 2-3 tháng
\end{itemize}

\subsection*{Mẹo Bảo Quản}
\begin{itemize}[leftmargin=1.5em]
    \item Luôn ghi ngày tháng lên túi/hộp đựng thức ăn
    \item Ép hết không khí trong túi zip trước khi đông lạnh
    \item Không để thức ăn nóng trực tiếp vào tủ lạnh (để nguội trước)
    \item Rã đông thịt trong ngăn mát hoặc ngâm nước lạnh, không dùng nước nóng
    \item Không nên rã đông rồi đông lại nhiều lần
\end{itemize}

\vspace{1cm}

\section*{Thuật Ngữ Nấu Ăn Cơ Bản}

\begin{description}[leftmargin=6em, style=nextline]
    \item[Xào] Nấu nhanh trên lửa lớn với ít dầu, đảo liên tục
    \item[Chiên] Nấu trong nhiều dầu ở nhiệt độ cao
    \item[Luộc] Nấu trong nước sôi
    \item[Hầm/Kho] Nấu lâu ở nhiệt độ thấp với ít nước
    \item[Nướng] Nấu bằng nhiệt khô (lò nướng, air fryer, vỉ nướng)
    \item[Hấp] Nấu bằng hơi nước
    \item[Chần] Luộc nhanh trong nước sôi (vài phút)
    \item[Ướp] Ngâm thịt/rau với gia vị để ngấm vị
    \item[Phi thơm] Làm nóng dầu, thêm tỏi/hành cho thơm
    \item[Nêm nếm] Thêm gia vị và thử vị
    \item[Rã đông] Làm tan đá thực phẩm đông lạnh
    \item[Thái hạt lựu] Thái thành miếng nhỏ vuông
    \item[Thái sợi] Thái thành dạng sợi dài, mỏng
    \item[Thái lát] Thái thành lát mỏng
    \item[Băm] Thái rất nhỏ, gần như nhuyễn
    \item[Đập dập] Dùng mặt phẳng dao đập nhẹ (với tỏi, gừng, sả)
\end{description}

\vspace{1cm}

\section*{Kỹ Thuật An Toàn}

\subsection*{Khi Sử Dụng Dao}
\begin{itemize}[leftmargin=1.5em]
    \item Luôn giữ dao sắc (dao tù dễ trượt gây nguy hiểm hơn)
    \item Cuộn ngón tay khi cầm thực phẩm, không để ngón tay thẳng
    \item Thái trên thớt ổn định, không để thớt trơn trượt
    \item Không để dao trong bồn rửa có bọt
    \item Rửa dao cầm cán, lau từ gốc đến ngọn
\end{itemize}

\subsection*{Khi Nấu Nướng}
\begin{itemize}[leftmargin=1.5em]
    \item Tay cầm nồi/chảo luôn quay ra ngoài bếp
    \item Không để trẻ em gần bếp khi đang nấu
    \item Dùng găng tay/khăn khi cầm đồ nóng
    \item Tắt bếp khi rời khỏi bếp
    \item Có bình chữa cháy hoặc chăn chữa cháy sẵn sàng
    \item Nếu dầu bốc cháy, đậy nắp lại, tắt bếp (KHÔNG dùng nước dập lửa)
\end{itemize}

\subsection*{Vệ Sinh An Toàn Thực Phẩm}
\begin{itemize}[leftmargin=1.5em]
    \item Rửa tay trước và sau khi chạm vào thực phẩm sống
    \item Dùng thớt riêng cho thịt sống và thực phẩm chín
    \item Rửa rau củ kỹ, ngâm nước muối loãng 5-10 phút
    \item Nấu chín kỹ thịt, đặc biệt là thịt gà, heo
    \item Không để thức ăn chín ngoài nhiệt độ phòng quá 2 giờ
\end{itemize}

\vspace{1cm}

\section*{Mẹo Khắc Phục Lỗi Thường Gặp}

\begin{description}[leftmargin=3em, style=nextline]
    \item[Món ăn quá mặn] \vspace{-0.5em}
        \begin{itemize}
            \item Thêm nước/nước dừa để pha loãng
            \item Thêm đường để cân bằng vị
            \item Thêm khoai tây thái miếng vào nấu (khoai sẽ hút muối)
        \end{itemize}

    \item[Món ăn quá chua] \vspace{-0.5em}
        \begin{itemize}
            \item Thêm đường từ từ cho đến khi cân bằng
            \item Thêm chút muối để trung hòa
        \end{itemize}

    \item[Món ăn quá cay] \vspace{-0.5em}
        \begin{itemize}
            \item Thêm đường và nước
            \item Thêm sữa chua hoặc sữa tươi
            \item Thêm thêm nguyên liệu chính để pha loãng
        \end{itemize}

    \item[Cơm bị nhão] \vspace{-0.5em}
        \begin{itemize}
            \item Trải cơm ra đĩa rộng, để nguội, dùng quạt thổi cho khô
            \item Có thể dùng làm cháo hoặc cơm chiên
        \end{itemize}

    \item[Thịt bị khô/chai] \vspace{-0.5em}
        \begin{itemize}
            \item Lần sau giảm thời gian nấu
            \item Ướp thịt trước khi nấu để mềm hơn
            \item Dùng nước/nước dừa kho lại cho mềm
        \end{itemize}

    \item[Rau bị nhũn/vàng] \vspace{-0.5em}
        \begin{itemize}
            \item Lần sau xào lửa lớn hơn, thời gian ngắn hơn
            \item Chần rau qua nước sôi rồi ngâm nước đá để giữ màu
        \end{itemize}

    \item[Canh/soup quá loãng] \vspace{-0.5em}
        \begin{itemize}
            \item Nấu tiếp để nước cạn bớt
            \item Hòa 1 muỗng bột năng với nước lạnh, rót vào canh khuấy đều
        \end{itemize}
\end{description}

\vspace{1cm}

\section*{Thay Thế Nguyên Liệu}

\begin{description}[leftmargin=6em, style=nextline]
    \item[Nước mắm] $\rightarrow$ Nước tương + muối
    \item[Nước dừa] $\rightarrow$ Nước lọc + chút sữa tươi
    \item[Gừng tươi] $\rightarrow$ Gừng khô nghiền (giảm 1/2 lượng)
    \item[Tỏi tươi] $\rightarrow$ Bột tỏi khô (1 tép tỏi = 1/4 muỗng cà phê bột)
    \item[Chanh tươi] $\rightarrow$ Giấm (giảm 1/2 lượng)
    \item[Đường trắng] $\rightarrow$ Đường nâu, mật ong (cùng lượng)
    \item[Sữa tươi] $\rightarrow$ Sữa đặc + nước (pha loãng)
    \item[Thịt gà] $\leftrightarrow$ Thịt heo (có thể thay thế cho nhau trong hầu hết công thức)
\end{description}

\vspace{1cm}

\textit{Với những kiến thức cơ bản này, bạn đã sẵn sàng bước vào bếp và thực hành các công thức trong sổ tay. Hãy bắt đầu từ những món đơn giản và dần dần nâng cao kỹ năng!}
