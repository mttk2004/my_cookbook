\documentclass[a5paper,11pt,twoside]{book}

% --- CẤU HÌNH KHỔ GIẤY VÀ CĂN LỀ ---
\usepackage[
    a5paper,
    inner=2.5cm,
    outer=1.0cm, % Tôi tăng nhẹ lên 1cm để nội dung đỡ bị sát mép giấy quá khi cầm
    top=1.5cm,   % Tăng lề trên một chút để header thoáng hơn
    bottom=1.5cm
]{geometry}

% --- CẤU HÌNH TIẾNG VIỆT VÀ FONT CHỮ ---
\usepackage{fontspec}
\usepackage[vietnamese]{babel}

\setmainfont{FreeSerif}
\setsansfont{FreeSans}
\setmonofont{FreeMono}

% --- CÁC GÓI TIỆN ÍCH ---
\usepackage{graphicx}
\usepackage{xcolor}
\usepackage{titlesec}
\usepackage{fancyhdr}
\usepackage[hidelinks]{hyperref}
\usepackage{enumitem} % Để tùy chỉnh list (itemize/enumerate) đẹp hơn

% --- TRANG TRÍ HEADER/FOOTER ---
\pagestyle{fancy}
\fancyhf{}
\fancyhead[LE,RO]{\textbf{\thepage}}
\fancyhead[RE]{\itshape\nouppercase{\leftmark}}
\fancyhead[LO]{\itshape\nouppercase{\rightmark}}

% --- ĐỊNH NGHĨA GIAO DIỆN MÓN ĂN (MACRO) ---
% Cú pháp: \monan{Tên món}{Nguyên liệu}{Cách làm}
\newcommand{\monan}[3]{
    \section{#1} % Tên món sẽ vào Mục lục
    \vspace{0.3cm}
    \textbf{\large \color{brown!80!black} $\bullet$ Nguyên liệu:}
    \begin{itemize}[leftmargin=1.5em, labelsep=0.5em]
        #2
    \end{itemize}
    \vspace{0.3cm}
    \textbf{\large \color{brown!80!black} $\bullet$ Chế biến:}
    \begin{enumerate}[leftmargin=1.5em, labelsep=0.5em]
        #3
    \end{enumerate}
    \vspace{0.5cm}
    \hrule
    \vspace{0.5cm}
}

% --- TRANG TRÍ TIÊU ĐỀ CHƯƠNG ---
\titleformat{\chapter}[display]
  {\normalfont\bfseries\color{brown!60!black}}
  {\filleft\huge\chaptertitlename\ \thechapter}
  {1ex}
  {\titlerule\vspace{1ex}\filleft\Huge}
  [\vspace{1ex}\titlerule]

% --- THÔNG TIN SÁCH ---
\title{\textbf{SỔ TAY CÔNG THỨC NẤU ĂN\\ \large (Dành cho người bận rộn)}}
\author{Biên soạn: Gemini & User}
\date{\today}

% ================= NỘI DUNG CHÍNH =================
\begin{document}

\frontmatter
\maketitle
% \tableofcontents
\mainmatter

% Nhúng các file con vào (Lưu ý: tạo folder 'chapters' trước)
\chapter{Kỹ Thuật Meal Prep \& Sốt Ướp}

\section*{Quy trình trữ đông thịt}
Để tiết kiệm thời gian, hãy sơ chế thịt ngay khi mua về:
\begin{enumerate}
    \item Rửa sạch, thấm khô nước.
    \item Cắt thái theo món (thái lát mỏng, cắt khối, hoặc băm).
    \item Ướp sốt theo công thức bên dưới.
    \item Chia vào hộp/túi nhỏ (đủ 1 bữa ăn).
    \item Cấp đông.
\end{enumerate}
\textit{Lưu ý: Chuyển thịt từ ngăn đông xuống ngăn mát vào tối hôm trước để rã đông từ từ.}

\section{Các Công Thức Sốt Ướp Cơ Bản}

\subsection*{1. Sốt Ướp Kho/Rim}
\textbf{Dùng cho:} Thịt heo cắt khối, gà chặt miếng (Gà kho gừng, Heo kho tiêu).
\begin{itemize}
    \item 2 thìa nước mắm
    \item 1 thìa đường
    \item 1 thìa tỏi băm
    \item 1/2 thìa tiêu
    \item Ớt bột/ớt tươi (tùy thích)
\end{itemize}

\subsection*{2. Sốt Ướp Xào Mềm}
\textbf{Dùng cho:} Thịt heo thái lát mỏng, ức gà thái lát/hạt lựu.
\begin{itemize}
    \item 1 thìa nước tương
    \item 1 thìa dầu hào (hoặc thay bằng 1/2 thìa đường + xíu muối)
    \item 1 thìa tỏi băm
    \item \textbf{1 thìa dầu ăn} (Quan trọng: giúp thịt mềm khi đông đá)
\end{itemize}

\subsection*{3. Sốt Ướp Sả Ớt}
\textbf{Dùng cho:} Gà, thịt heo ba chỉ.
\begin{itemize}
    \item Sả băm + Ớt băm
    \item 2 thìa nước mắm
    \item 1 thìa đường
    \item Chút bột nghệ (nếu có)
\end{itemize}
\chapter{Món Ngon Với Nồi Cơm Điện}

\monan{Cơm Gà Hải Nam (Sinh Viên)}
{
    \item Gạo tẻ
    \item Thịt gà (đùi hoặc ức)
    \item Gừng, tỏi băm nhỏ
    \item Hành tây (nếu có)
}
{
    \item Vo gạo sạch, bỏ vào nồi.
    \item Trộn trực tiếp 1 thìa dầu ăn, tỏi băm, gừng băm vào gạo (giúp cơm thơm và bóng).
    \item Xếp thịt gà lên trên mặt gạo. Đổ nước vừa đủ như nấu cơm bình thường.
    \item Bấm nút \textbf{Cook}.
    \item Khi chín, lấy gà ra xé hoặc chặt. Ăn kèm nước tương tỏi ớt và dưa leo.
}

\monan{Cơm Trộn Thập Cẩm (Dọn Tủ Lạnh)}
{
    \item Gạo
    \item Xúc xích (cắt hạt lựu)
    \item Rau củ đông lạnh hỗn hợp (ngô, cà rốt, đậu hà lan)
    \item Gia vị: Muối, nước tương
}
{
    \item Vo gạo, cho nước vào nồi.
    \item Cho xúc xích, rau củ đông lạnh vào chung với gạo.
    \item Nêm vào nước nấu cơm 1 chút muối và 1 thìa nước tương.
    \item Bấm nút \textbf{Cook}.
    \item Khi chín, có thể đập thêm 1 quả trứng vào trộn đều cùng tương ớt.
}
\chapter{Món Xào \& Luộc (Ít Dầu)}

\monan{Gà Xào Sả Ớt}
{
    \item Thịt gà (đã ướp sốt Sả Ớt ở Chương 1)
    \item Tỏi, hành tím
    \item Nước mắm, đường
}
{
    \item Phi thơm tỏi. Cho gà vào đảo săn ở lửa lớn.
    \item Gà săn lại thì hạ lửa, thêm xíu nước nếu quá khô.
    \item Đảo đến khi nước sốt sệt lại bám quanh miếng gà.
    \item Tắt bếp, rắc thêm chút tiêu.
}

\monan{Thịt Heo Xào Rau Củ/Đậu}
{
    \item Thịt heo (đã ướp sốt Xào ở Chương 1)
    \item Đậu cove/Đậu bắp/Cải thìa (cắt khúc)
    \item Tỏi đập dập
}
{
    \item Phi tỏi thơm, cho thịt heo vào xào chín tới rồi trút ra đĩa riêng (để thịt không bị dai).
    \item Dùng chảo cũ, cho rau củ vào xào. Thêm chút nước để rau chín đều mà không cháy. Nêm xíu hạt nêm/muối.
    \item Khi rau chín, đổ thịt heo vào đảo chung 30 giây rồi tắt bếp.
}

\monan{Rau Củ Luộc Kho Quẹt}
{
    \item Kho quẹt (có sẵn)
    \item Các loại rau: Đậu bắp, cà rốt, cải thìa, rau muống
    \item Muối tinh
}
{
    \item Đun sôi nồi nước, thêm 1 nhúm muối (giúp rau xanh).
    \item Luộc các loại củ (cà rốt) trước vì lâu chín, sau đó đến rau lá.
    \item Rau chín vớt ra để ráo.
    \item Hâm nóng kho quẹt và chấm kèm.
}
\chapter{Món Canh Nấu Nhanh}

\monan{Canh Trứng Cà Chua (5 Phút)}
{
    \item 2 quả cà chua (cắt múi cau)
    \item 1-2 quả trứng gà
    \item Hành lá, hành tím
}
{
    \item Phi thơm hành tím, xào cà chua chín mềm (nêm xíu muối để cà chua nhanh nát và lên màu đỏ đẹp).
    \item Đổ nước vào đun sôi.
    \item Đập trứng ra bát, đánh tan. Đổ từ từ trứng vào nồi nước đang sôi sùng sục, vừa đổ vừa lấy đũa khuấy nhẹ để tạo vân mây.
    \item Nêm nước mắm/gia vị vừa ăn. Rắc hành lá và tắt bếp.
}

\monan{Canh Rau Thịt Băm}
{
    \item Rau mồng tơi/Rau muống/Cải thìa
    \item Thịt heo băm (hoặc cắt nhỏ)
}
{
    \item Đun sôi nước. Cho thịt băm vào nấu chín, hớt bỏ bọt cho nước trong.
    \item Thả rau vào.
    \item Chờ nước sôi bùng lại, rau vừa chín tới thì nêm gia vị rồi tắt bếp ngay (để rau giữ độ giòn và vitamin).
}

\end{document}