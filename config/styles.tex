% ================================================================
% ĐỊNH NGHĨA MACRO VÀ STYLE (CUSTOM MACROS & STYLES)
% ================================================================
% File này chứa các macro tùy chỉnh và định nghĩa style cho dự án

% --- Tạo counter riêng cho món ăn ---
\newcounter{dishcounter}
\counterwithin*{dishcounter}{chapter}

% --- ĐỊNH NGHĨA GIAO DIỆN MÓN ĂN (MACRO) ---
% Cú pháp: \monan{Tên món}{Độ khó}{Thời gian}{Nguyên liệu}{Sơ chế}{Chế biến}
%
% Tham số:
%   #1 - Tên món (sẽ hiển thị trong mục lục)
%   #2 - Độ khó (1=dễ, 2=trung bình, 3=khó)
%   #3 - Thời gian (ví dụ: "15 phút", "30 phút")
%   #4 - Danh sách nguyên liệu (dùng \item cho mỗi nguyên liệu)
%   #5 - Các bước sơ chế (để trống nếu không có, dùng \item cho mỗi bước)
%   #6 - Các bước chế biến (dùng \item cho mỗi bước)
%
% Ví dụ:
%   \monan{Thịt kho tàu}{2}{45 phút}{
%       \item 500g thịt ba chỉ
%       \item 2 quả trứng
%   }{
%       \item Luộc thịt sơ qua
%   }{
%       \item Kho thịt với nước dừa
%   }
\newcommand{\monan}[6]{
    \refstepcounter{dishcounter}
    \addcontentsline{toc}{section}{\protect\numberline{\thedishcounter}#1}
    \index{#1} % Thêm món ăn vào chỉ mục

    % Tiêu đề món ăn và thông tin độ khó/thời gian trên cùng một dòng
    \noindent
    \begin{minipage}[t]{0.7\textwidth}
        {\Large\bfseries\color{brown!80!black}\thedishcounter. #1}
    \end{minipage}%
    \hfill
    \begin{minipage}[t]{0.27\textwidth}
        \raggedleft
        \textcolor{brown!70!black}{
            \ifcase#2
            \or $\star$
            \or $\star\star$
            \or $\star\star\star$
            \fi
            \quad $\bullet$ \quad
            #3
        }
    \end{minipage}

    \vspace{0.4cm}

    % Phần nguyên liệu
    \textbf{\normalsize \color{brown!80!black} $\bullet$ Nguyên liệu:}
    \begin{itemize}[leftmargin=1.2em, labelsep=0.3em, itemsep=0pt, parsep=0pt]
        #4
    \end{itemize}
    \vspace{0.2cm}

    % Phần sơ chế (chỉ hiển thị nếu không rỗng)
    \ifx&#5&%
    \else
        \textbf{\normalsize \color{brown!80!black} $\bullet$ Sơ chế:}
        \begin{enumerate}[leftmargin=1.2em, labelsep=0.3em, itemsep=0pt, parsep=0pt]
            #5
        \end{enumerate}
        \vspace{0.2cm}
    \fi

    % Phần chế biến
    \textbf{\normalsize \color{brown!80!black} $\bullet$ Chế biến:}
    \begin{enumerate}[leftmargin=1.2em, labelsep=0.3em, itemsep=0pt, parsep=0pt]
        #6
    \end{enumerate}
    \vspace{0.5cm}

    % Phần ghi chú và đánh giá
    \noindent
    \begin{tikzpicture}
        \node[
            rectangle,
            draw=brown!30,
            line width=0.5pt,
            rounded corners=2pt,
            fill=brown!5,
            text width=\dimexpr\textwidth-1.5cm\relax,
            inner sep=0.4cm
        ] {
            \begin{minipage}{\dimexpr\textwidth-2cm\relax}
                \textcolor{brown!70!black}{\textbf{\small Ghi Chú Của Bạn:}}

                % TODO: Khoảng cách trước dòng đầu tiên (mặc định: 0.4cm)
                \vspace{0.4cm}
                \noindent
                % TODO: Màu sắc (brown!40), độ đậm (line width=0.8pt)
                \tikz \draw[brown!40, dashed, line width=0.8pt] (0,0) -- (\linewidth,0);
                % TODO: Khoảng cách giữa các dòng (mặc định: 0.5cm)
                \vspace{0cm}

                \noindent
                \tikz \draw[brown!40, dashed, line width=0.8pt] (0,0) -- (\linewidth,0);
                \vspace{0cm}

                \noindent
                \tikz \draw[brown!40, dashed, line width=0.8pt] (0,0) -- (\linewidth,0);

                % TODO: Khoảng cách sau dòng cuối (mặc định: 0.5cm)
                \vspace{0.2cm}
                \noindent
                \textcolor{brown!70!black}{\textbf{\small Đánh Giá:}}
                \quad
                $\star$ $\star$ $\star$ $\star$ $\star$
                \quad\quad
                \textbf{\small Đã thử:} $\square$
            \end{minipage}
        };
    \end{tikzpicture}

    \vspace{0.6cm}
}

% --- TRANG TRÍ TIÊU ĐỀ CHƯƠNG ---
% Tùy chỉnh giao diện cho \chapter
% Căn phải cho trang lẻ, căn trái cho trang chẵn
\titleformat{\chapter}[display]
  {\normalfont\bfseries\color{brown!60!black}}
  {\ifodd\value{page}\filleft\else\filright\fi\Large\chaptertitlename\ \thechapter}
  {1ex}
  {\ifodd\value{page}\filleft\else\filright\fi\huge\MakeUppercase}
  [\vspace{-1ex}]

% Điều chỉnh khoảng cách trên và dưới của chapter
% \titlespacing{command}{left}{before-sep}{after-sep}[right-sep]
\titlespacing{\chapter}
  {0pt}      % left margin
  {0pt}      % khoảng cách trước (bỏ khoảng trống trên)
  {3ex}      % khoảng cách sau (tăng khoảng trống dưới một chút)
