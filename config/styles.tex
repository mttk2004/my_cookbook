% ================================================================
% ĐỊNH NGHĨA MACRO VÀ STYLE (CUSTOM MACROS & STYLES)
% ================================================================
% File này chứa các macro tùy chỉnh và định nghĩa style cho dự án

% --- ĐỊNH NGHĨA GIAO DIỆN MÓN ĂN (MACRO) ---
% Cú pháp: \monan{Tên món}{Nguyên liệu}{Sơ chế}{Chế biến}
%
% Tham số:
%   #1 - Tên món (sẽ hiển thị trong mục lục)
%   #2 - Danh sách nguyên liệu (dùng \item cho mỗi nguyên liệu)
%   #3 - Các bước sơ chế (để trống nếu không có, dùng \item cho mỗi bước)
%   #4 - Các bước chế biến (dùng \item cho mỗi bước)
%
% Ví dụ:
%   \monan{Thịt kho tàu}{
%       \item 500g thịt ba chỉ
%       \item 2 quả trứng
%   }{
%       \item Luộc thịt sơ qua
%   }{
%       \item Kho thịt với nước dừa
%   }
\newcommand{\monan}[4]{
    \section{#1} % Tên món sẽ vào Mục lục
    \vspace{0.2cm}

    % Phần nguyên liệu
    \textbf{\normalsize \color{brown!80!black} $\bullet$ Nguyên liệu:}
    \begin{itemize}[leftmargin=1.2em, labelsep=0.3em, itemsep=0pt, parsep=0pt]
        #2
    \end{itemize}
    \vspace{0.2cm}

    % Phần sơ chế (chỉ hiển thị nếu không rỗng)
    \ifx&#3&%
    \else
        \textbf{\normalsize \color{brown!80!black} $\bullet$ Sơ chế:}
        \begin{enumerate}[leftmargin=1.2em, labelsep=0.3em, itemsep=0pt, parsep=0pt]
            #3
        \end{enumerate}
        \vspace{0.2cm}
    \fi

    % Phần chế biến
    \textbf{\normalsize \color{brown!80!black} $\bullet$ Chế biến:}
    \begin{enumerate}[leftmargin=1.2em, labelsep=0.3em, itemsep=0pt, parsep=0pt]
        #4
    \end{enumerate}
    \vspace{0.3cm}

    % Đường phân cách giữa các món (kiểu đường chấm + trái tim)
    \vspace{0.4cm}
    \begin{center}
        \textcolor{brown!60!black}{
            \dotfill \quad $\heartsuit$ \quad \dotfill
        }
    \end{center}
    \vspace{0.4cm}

}

% --- TRANG TRÍ TIÊU ĐỀ CHƯƠNG ---
% Tùy chỉnh giao diện cho \chapter
\titleformat{\chapter}[display]
  {\normalfont\bfseries\color{brown!60!black}}
  {\filleft\Large\chaptertitlename\ \thechapter}
  {2ex}
  {\filleft\huge\MakeUppercase}
  []
